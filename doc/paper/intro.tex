\section{Introduction}
% Putting the Automata Theory Withing Reach

\paragraph{}
Teaching abstract concepts is only possible if these concepts are put within student's reach. There are multiple ways to do this and each teacher has its own approach, as well as each student is more sensitive to some ways of knowledge transmission than to other ways. Some students might be more sensitive to the structure of the speech and the mathematical robustness of definitions, some might be more likely to assimilate knowledge if they can hear a teacher explain it, some might need to visualize concepts with their eyes, some might need a mix of several ways and the more ways are exploited to transmit things, the more student should be involved in the knowledge transmission as these ways are more complementary than redundant.


\paragraph{}
Teachers who are eager to maximize the number of ways they take to transmit knowledge might be interested in making abstract concepts visualizable, dynamic therefore adding a concrete aspect of the concepts. The goal of the tool is to put the automata theory in student's hands, making it manipulable (``hackable'').
